\documentclass[11pt]{article}
\usepackage{amsfonts,amsmath,amssymb,amsthm,graphicx}
\usepackage{multirow}
\usepackage{booktabs}
\usepackage[caption=false]{subfig}
\usepackage{color}
\usepackage{ifthen}

\setlength{\topmargin}{0 mm}
\setlength{\oddsidemargin}{5 mm}
\setlength{\evensidemargin}{5 mm}
\setlength{\textwidth}{150 mm}
\setlength{\textheight}{210 mm}

\newtheorem{lemma}{Lemma}
\newtheorem{theorem}{Theorem}
\newtheorem{remark}{Remark}
\newcommand{\dzx}{D_0^x}
\newcommand{\dzxp}{D_{0p}^x}
\newcommand{\wdzx}{\widetilde{D_0^x}}
\newcommand{\wdzxp}{\widetilde{D_{0p}^x}}
\newcommand{\dpx}{D_+^x}
\newcommand{\dmx}{D_-^x}
\newcommand{\dzy}{D_0^y}
\newcommand{\dzyp}{D_{0p}^y}
\newcommand{\wdzy}{\widetilde{D_0^y}}
\newcommand{\dpy}{D_+^y}
\newcommand{\dmy}{D_-^y}
\newcommand{\dzz}{D_0^z}
\newcommand{\wdzz}{\widetilde{D_0^z}}
\newcommand{\dpz}{D_+^z}
\newcommand{\dmz}{D_-^z}
\newcommand{\dzt}{D_0^t}
\newcommand{\dpt}{D_+^t}
\newcommand{\dmt}{D_-^t}
\newcommand{\ehx}{E_{1/2}^x}
\newcommand{\ehy}{E_{1/2}^y}
\newcommand{\ehz}{E_{1/2}^z}
%
\newcommand{\calo}{{\cal O}}
%
\newcommand{\ab}{{\mathbf a}}
\newcommand{\bb}{{\mathbf b}}
\newcommand{\db}{{\mathbf d}}
\newcommand{\eb}{{\mathbf e}}
\newcommand{\fb}{{\mathbf f}}
\newcommand{\gb}{{\mathbf g}}
\newcommand{\ib}{{\mathbf i}}
\newcommand{\jb}{{\mathbf j}}
\newcommand{\nb}{{\mathbf n}}
\newcommand{\pb}{{\mathbf p}}
\newcommand{\tb}{{\mathbf t}}
\newcommand{\rb}{{\mathbf r}}
\newcommand{\yb}{{\mathbf y}}
\newcommand{\zb}{{\mathbf z}}
\newcommand{\qb}{{\mathbf q}}
\newcommand{\ub}{{\mathbf u}}
\newcommand{\vb}{{\mathbf v}}
\newcommand{\wb}{{\mathbf w}}

\newcommand{\Ab}{{\mathbf A}}
\newcommand{\Bb}{{\mathbf B}}
\newcommand{\Eb}{{\mathbf E}}
\newcommand{\Fb}{{\mathbf F}}
\newcommand{\Ib}{{\mathbf I}}
\newcommand{\Hb}{{\mathbf H}}
\newcommand{\Kb}{{\mathbf K}}
\newcommand{\Lb}{{\mathbf L}}
\newcommand{\Pb}{{\mathbf P}}
\newcommand{\Qb}{{\mathbf Q}}
\newcommand{\Rb}{{\mathbf R}}
\newcommand{\Ub}{{\mathbf U}}
\newcommand{\Tb}{{\mathbf T}}
\newcommand{\Xb}{{\mathbf X}}

\newcommand{\Abb}{\mathbb{A}}
\newcommand{\Bbbb}{\mathbb{B}}
\newcommand{\Ebb}{\mathbb{E}}
\newcommand{\Fbb}{\mathbb{F}}
\newcommand{\Ibb}{\mathbb{I}}
\newcommand{\Hbb}{\mathbb{H}}
\newcommand{\Kbb}{\mathbb{K}}
\newcommand{\Lbb}{\mathbb{L}}
\newcommand{\Pbb}{\mathbb{P}}
\newcommand{\Qbb}{\mathbb{Q}}
\newcommand{\Rbb}{\mathbb{R}}
\newcommand{\Ubb}{\mathbb{U}}
\newcommand{\Tbb}{\mathbb{T}}
\newcommand{\Xbb}{\mathbb{X}}
\newcommand{\Ybb}{\mathbb{Y}}
\newcommand{\Zbb}{\mathbb{Z}}

\newcommand{\uh}{\hat{u}}
\newcommand{\vh}{\hat{v}}
\newcommand{\ph}{\hat{p}}
\newcommand{\qh}{\hat{q}}

\newcommand{\psib}{{\mathbf \psi}}
\newcommand{\Psib}{{\mathbf \Psi}}

\newcommand{\re}{{\rm Re}\,}
\newcommand{\im}{{\rm Im}\,}

\renewcommand{\arraystretch}{1.3}
%
\newcommand{\p}{\partial}
%
\newcommand{\eq}{\!\!\! = \!\!\!}
\newcommand{\om}{\omega}
%\newcommand{\divergence}{\nabla\cdot}
%\newcommand{\curl}{\nabla\times}

\newcommand{\rhob}{\boldsymbol{\rho}}
\newcommand{\kab}{\boldsymbol{\kappa}}
\newcommand{\etab}{\boldsymbol{\eta}}
\newcommand{\zetab}{\boldsymbol{\zeta}}
\newcommand{\sigmab}{\boldsymbol{\sigma}}
\newcommand{\omegab}{\boldsymbol{\omega}}
\newcommand{\Gb}{{\mathbf G}}
\newcommand{\kb}{{\mathbf k}}
\newcommand{\sbold}{{\mathbf s}}
\newcommand{\ba}{\begin{array}}
\newcommand{\ea}{\end{array}}
\newcommand{\be}{\begin{equation}}
\newcommand{\ee}{\end{equation}}
\newcommand{\bd}{\begin{displaymath}}
\newcommand{\ed}{\end{displaymath}}
\newcommand{\pa}{\partial}
\newcommand{\f}{\frac}
\newcommand{\drp}{D^r_+}
\newcommand{\drm}{D^r_-}
\newcommand{\dqp}{D^q_+}
\newcommand{\dqm}{D^q_-}
\newcommand{\dtqn}{\widetilde{{D^q_0}} }
\newcommand{\dtrn}{\widetilde{{D^r_0}} }
\newcommand{\dqn}{D^q_0}
\newcommand{\drn}{D^r_0}
\newcommand{\erh}{E^r_{1/2}}
\newcommand{\eqh}{E^q_{1/2}}

\def\dpl{D_+}
\def\dmi{D_-}

\newcommand{\ubbar}{\bar{\mathbf{u}}}
\newcommand{\ubar}{\bar{u}}

% Numerical solutions
% \newcommand{\vb}{\mathbf{v}}


% Grids
\newcommand{\xb}{\mathbf{x}}
\newcommand{\ybh}{\hat{\mathbf{x}}}
\newcommand{\xbh}{\hat{\mathbf{x}}}
\newcommand{\Ja}{J_{\alpha}}
\newcommand{\ga}{g_{\alpha}}
\newcommand{\Ma}{M_{\alpha}}

% Interpolation
\newcommand{\Nxy}{N_{\mathbf{x}\rightarrow\hat{\mathbf{x}}}}
\newcommand{\Nyx}{N_{\hat{\mathbf{x}}\rightarrow\mathbf{x}}}
\newcommand{\Nuv}{N_{\bar{\ub}_1\rightarrow\bar{\ub}_2}}
\newcommand{\Nvu}{N_{\bar{\ub}_2\rightarrow\bar{\ub}_1}}
\newcommand{\Nij}{N_{\bar{\ub}_i\rightarrow\bar{\ub}_j}}
\newcommand{\Nji}{N_{\bar{\ub}_j\rightarrow\bar{\ub}_i}}
\newcommand{\Px}{P}
\newcommand{\Pxh}{\hat{P}}

% Domains
\newcommand{\domp}{\Omega_{p}}
\newcommand{\domu}{\Omega_{\ub}}
\newcommand{\domv}{\Omega_{\vb}}
\newcommand{\domui}{\Omega_{\bar{\mathbf{u}}_i}}
\newcommand{\domuj}{\Omega_{\bar{\mathbf{u}}_j}}
\newcommand{\gamja}{\Gamma_{j0}}
\newcommand{\gamjb}{\Gamma_{j1}}
\newcommand{\gamia}{\Gamma_{i0}}
\newcommand{\gamib}{\Gamma_{i1}}

% Comments
\newcommand{\red}{\color{red} AP:}
\newcommand{\usecomments}{true}
\newcommand{\ocomment}[1] {
\ifthenelse{ \equal{\usecomments}{true} }{
 \textbf{Ossian: }{\color{blue} #1}
}{
}
}



\begin{document}

%% \title{Analytical Solution of two oscillator problem}

%% \author{N. Anders Petersson\thanks{Center for Applied
%%     Scientific Computing, Lawrence Livermore National Laboratory, L-561, PO Box 808, Livermore CA
%%     94551. }}

%% \date{\today}

%% \maketitle
\section{An analytical solution for two oscillator problem}
We wish to construct an analytical solution for a decoupled two-oscillator system.  We start by finding two separate solutions to one-oscillator systems and then use them to construct the solution to the two-oscillator case.  Consider the Schro\"odinger equation
\begin{equation}\label{eq_schrodinger}
\dot{\psi} = -i H(t) \psi,\quad \psi(0) = \psi_0.
\end{equation}
\textbf{Solution 1}: Suppose the Hamiltonian is given by
\begin{equation}
    H(t)=H_1(t) = f(t)(a+a^\dag), \quad a+a^\dag =
    \begin{bmatrix}
        0 & 1\\
        1 & 0
    \end{bmatrix}, \quad f(t) = \frac{1}{4}(1-\cos(\omega t))
\end{equation}
An analytical solution is given by
\begin{equation}
    \psi_1(t) = \begin{bmatrix}
                    \cos(\phi(t))\\
                    -i\sin(\phi(t))]
                \end{bmatrix}, \quad \quad \phi(t) = \frac{1}{4}(t-\frac{1}{\omega}\sin(\omega t))
\end{equation}
\textbf{Check:}
\begin{align}
    \dot{\psi_1} &= \frac{1}{4}(1-\cos(\omega t)) \begin{bmatrix} -\sin(\phi)\\
                                    -i\cos(\phi)) \end{bmatrix}\\
    -iH_1\psi_1 &= \frac{-i}{4}(1-\cos(\omega t))\begin{bmatrix} 0 & 1\\ 1 & 0 \end{bmatrix}
                                            \begin{bmatrix} \cos(\phi) \\ -i\sin(\phi) \end{bmatrix} =
                                            \frac{1}{4}(1-\cos(\omega t)) \begin{bmatrix} -\sin(\phi)\\
                                            -i\cos(\phi) \end{bmatrix}
\end{align}
\textbf{Solution 2}: Suppose the Hamiltonian is given by
\begin{equation}
    H(t)=H_2(t) = ig(t)(a-a^\dag), \quad a-a^\dag =
    \begin{bmatrix}
        0 & 1\\
        -1 & 0
    \end{bmatrix}, \quad g(t) = \frac{1}{4}(1-\sin(\omega t))
\end{equation}
An analytical solution is given by
\begin{equation}
    \psi_2(t) = \begin{bmatrix}
                    \cos(\theta(t))\\
                    -\sin(\theta(t))]
                \end{bmatrix}, \quad \quad \theta(t) = \frac{1}{4}(t+\frac{1}{\omega}\cos(\omega t)-1)
\end{equation}
\textbf{Check:}
\begin{align}
    \dot{\psi_2} &= \frac{1}{4}(1-\sin(\omega t)) \begin{bmatrix} -\sin(\theta)\\
                                    -cos(\theta)) \end{bmatrix}\\
    -iH_2\psi_1 &= \frac{1}{4}(1-\sin(\omega t))\begin{bmatrix} 0 & 1\\ -1 & 0 \end{bmatrix}
                                            \begin{bmatrix} \cos(\theta) \\ -\sin(\theta) \end{bmatrix} =
                                            \frac{1}{4}(1-\sin(\omega t)) \begin{bmatrix} -\sin(\theta)\\
                                            -\cos(\theta) \end{bmatrix}
\end{align}
\textbf{Constructing the two-oscillator solution:}
We now show that the solutions 1 and 2 can be used to solve eq \eqref{eq_schrodinger} for the two-oscillator system given that $\psi = \psi_1 \otimes \psi_2$ and the Hamiltonian is described as $H(t) = H_1(t)\otimes I_2 + I_2 \otimes H_2(t)$ where $I_2$ is the $2\times2$ identity matrix.
\begin{equation}\label{psidot}
    \dot{\psi} = \dot{\psi_1}\otimes \psi_2 + \psi_1 \otimes \dot{\psi_2} = -i(H_1\psi_1\otimes \psi_2
                    + \psi_1\otimes H_2\psi_2)
\end{equation}
\begin{align}
    H_1\psi_1 \otimes \psi_2 = f(t)\begin{bmatrix} -i\sin(\phi)\\ \cos(\phi) \end{bmatrix} \otimes
                                   \begin{bmatrix} \cos(\theta)\\ -\sin(\theta) \end{bmatrix} =
                               f(t)\begin{bmatrix} -i\sin(\phi)\cos(\theta)\\ i\sin(\phi)\sin(\theta)\\
                                                   \cos(\phi)\cos(\theta)\\ -\cos(\phi)\sin(\theta)\end{bmatrix}\\
    \psi_1 \otimes H_2\psi_2 = g(t)\begin{bmatrix} \cos(\phi)\\ -i\sin(\phi) \end{bmatrix} \otimes
                                    \begin{bmatrix}-i\sin(\theta)\\ -i\cos(\theta) \end{bmatrix} =
                                -g(t)\begin{bmatrix} i\cos(\phi)\sin(\theta)\\ i\cos(\phi)\cos(\theta)\\
                                                    \sin(\phi)\sin(\theta)\\\sin(\phi)\cos(\theta)\end{bmatrix}
\end{align}
where we have substituted solutions 1 and 2 in the second equality of \eqref{psidot}.  Now looking at the right hand side:
\begin{equation}
    -iH\psi = -i\begin{bmatrix} 0 & ig(t) & f(t) & 0\\
                                 -ig(t) & 0 & 0 & f(t)\\
                                f(t) & 0 & 0 & ig(t)\\
                                0 & f(t) & -ig(t) & 0\end{bmatrix}
                \begin{bmatrix} \cos(\phi)\cos(\theta)\\
                                -\cos(\phi)\sin(\theta)\\
                                -i\sin(\phi)\cos(\theta)\\
                                i\sin(\phi)\sin(\theta)\end{bmatrix}
\end{equation}
it holds that $-i(H_1\psi_1\otimes \psi_2 + \psi_1\otimes H_2\psi_2) = -i(H_1\otimes I_2 + I_2 \otimes H_2)(\psi_1 \otimes \psi_2)$ and therefore $\dot{\psi} = -iH\psi$.\\\\
\textbf{Density Matrix:}
From here we can construct the corresponding density matrix
\begin{align}
    \rho(t) &= \psi(t)\psi^\dag(t), \quad \psi^\dag = \begin{bmatrix} \cos(\phi)\cos(\theta)\\-\cos(\phi)\sin(\theta)\\i\sin(\phi)\cos(\theta)\\-i\sin(\phi)\sin(\theta) \end{bmatrix}\\
    \rho(t) &= \begin{bmatrix}
                \cos^2(\phi)\cos^2(\theta) & & & \\
                & \cos^2(\phi)\sin^2(\theta) & & \\
                & & \sin^2(\phi)\cos^2(\theta) & \\
                & & & \sin^2(\phi)\sin^2(\theta)
            \end{bmatrix}
\end{align}



\end{document}
